\chapter{How to use this manual}

\section{What it is about}
This manual will contain tutorials and instructions on how to set up models in OGS6. It will contain information on how to write input files, which keywords to use, an index of these keywords and reference results.

\section{Adding authors}
Anyone who adds content to this file should add their name and affiliation to the list in the main file at which point they will automatically appear on the title page. The format to include yourself is 
\begin{verbatim}
	\author[0815]{Smart McGenius}
	\affil[0815]{Gewandhaus Leipzig}
\end{verbatim}

\section{Adding content}
If you add content, create a subdirectory with an appropriate name (e.g. process type or name of the numerical algorithm) and use \texttt{input} in the main file to include your content.

\section{Keyword indexing}
If you refer to OGS keywords, please add them to the index by using the \texttt{\\index} command:
\begin{verbatim}
	The porosity is given by $POROSITY \index{porosity}.
\end{verbatim}
Note that the index should be given without any \texttt{\$} sign.

\section{References}
If you refer to published material reference it by including the appropriate *.bib file.

\section{Figures and labelling}
Figures should be provided as pdf or png files. Any labels given to figures and equations should be unique so as to avoid overlapping label definitions. We suggest to include a process acronym and author initials into the label to facilitate this, e.g.
\begin{verbatim}
	\label{eq:THM_NW_strain}
	\label{fig:GF_NB_layers}
\end{verbatim}

\section{Compilation}
The document is to be compiled with pdflatex:
\begin{verbatim}
	pdflatex Theory_Manual.tex
\end{verbatim}
The bibliography can be built with 
\begin{verbatim}
	bibtex Theory_Manual
\end{verbatim}
The index, finally, is built with:
\begin{verbatim}
	makeindex Theory_Manual
\end{verbatim}